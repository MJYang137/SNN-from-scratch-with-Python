\documentclass[11pt,dvipdfmx,b5paper,oneside]{jsbook}

\usepackage{graphicx}
\usepackage{color}
\usepackage{here}
\usepackage{framed}
\usepackage{tcolorbox}
\usepackage{quotchap}
\usepackage{pdfpages}
\usepackage[hidelinks]{hyperref}
\usepackage{pxjahyper}
\usepackage{titlesec}
\usepackage{picture}
\usepackage{tikz}
\usepackage{amsmath,amssymb}
\usepackage{amsmath}
\usepackage{amsfonts}

\tcbuselibrary{breakable}
\definecolor{shadecolor}{gray}{0.80}

% section
\titleformat{\section}[block]{}{}{0pt}
{
  \definecolor{teal}{gray}{0.30}
  \begin{picture}(0,0)
    \put(-10,-5){
      \begin{tikzpicture}
        \fill[teal] (0pt,0pt) rectangle (5pt,19pt);
      \end{tikzpicture}
    }
    \put(-10,-5){
      \color{teal}
      \line(1,0){\hsize}
    }
  \end{picture}
  \hspace{0pt}
  \sf \Large \thesection
  \hspace{0pt}
}

% 図表見出し
\renewcommand{\tablename}{\textcolor{gray}{▼} 表}
\renewcommand{\figurename}{\textcolor{gray}{▲} 図}

\begin{document}

\chapter{脳のモデルとしての\\ニューラルネットワーク}
\section{Spiking model}
ホジキンハクスレーの式
\subsection{LIF}

\section{発火率モデル}
いよいよNNを見ていきます。私の話で恐縮ですが、NNを正しく脳のモデルと理解するまでかなり時間がかかりました。混乱した原因は「加算」と「閾値を超えると発火」という点です。結論から先に言うと、今言われている典型的なニューラルネットワークは「発火率モデル」というものです。\par
まず、初めにマカロック-ピッツ(McCulloch-Pitts)から見てみます。
ANNは脳のようであると言われているが実際にはそうではない。
発火率モデルとして捉えられる。ので、ミクロに見ればスパイクを出してはいないが、少しマクロにニューロンの活動のダイナミクスを見ると一致している。

\subsection{RNN}
\subsubsection{RNN}
微分方程式で書くと、
$$
\tau \frac{d \boldsymbol{r}}{dt}=-\boldsymbol{r}+f(W^{\text{rec}}\boldsymbol{r}+W^{\text{in}}\boldsymbol{u}+\boldsymbol{b}^{\text{rec}}+\boldsymbol{\xi})
$$
これを first-order Euler approximationを用いて離散化(time step $\Delta t$)すると、
$$
\boldsymbol{r}_t=(1-\alpha) \boldsymbol{r}_{t-1}+\alpha f(W^{\text{rec}}\boldsymbol{r}_{t-1}+W^{\text{in}}u_{t}+\boldsymbol{b}^{\text{rec}}+\boldsymbol{\xi}_t)
$$


\subsubsection{Daleの原理を守ったRNN}
Daleの法則.この法則は現在は修正されていますが、それでも

\section{これはsection}
我輩は猫である\footnote{こんな感じで脚注を書く}。

どこで生れたかとんと見当がつかぬ。何でも薄暗いじめじめした所でニャーニャー泣いていた事だけは記憶している。吾輩はここで始めて人間というものを見た。しかもあとで聞くとそれは書生という人間中で一番獰悪な種族であったそうだ。この書生というのは時々我々を捕えて煮て食うという話である。

\begin{tcolorbox}[breakable]
\begin{verbatim}
1  /* ここにはソースコードを書く */
2  #include<stdio.h>
3
4  int main(void)
5  {
6    printf("Hello, World!\n");
7    return 0;
8  }
9  /* breakableを付けるとこんな感じで改行にも対応できる */
\end{verbatim}
\end{tcolorbox}

\begin{shaded}
\begin{verbatim}
## ここにはコマンドを書く
$ echo "Hello, World!"
\end{verbatim}
\end{shaded}

図表はキャプションを付けたときに、先頭に「▲」や「▼」を付けるようにした。

\begin{table}[H]
  \centering
  \caption{表のサンプル}
  \begin{tabular}{|c|l|l|l|} \hline
    日本 & hoge & fuga & piyo \\ \hline
    アメリカ & foo & bar & baz \\ \hline
  \end{tabular}
  \label{table-sample}
\end{table}

\begin{tcolorbox}[title=これはコラム]
  コラムも随時挟めるようにした。

  tcolorboxはtitleを指定するといい感じにタイトル付きの枠で囲ってくれる。
\end{tcolorbox}

\end{document}